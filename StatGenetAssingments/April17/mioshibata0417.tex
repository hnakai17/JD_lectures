\documentclass[11pt, oneside]{article}   	% use "amsart" instead of "article" for AMSLaTeX format
\usepackage{geometry}                		% See geometry.pdf to learn the layout options. There are lots.
\geometry{letterpaper}                   		% ... or a4paper or a5paper or ... 
%\geometry{landscape}                		% Activate for rotated page geometry
%\usepackage[parfill]{parskip}    		% Activate to begin paragraphs with an empty line rather than an indent
\usepackage{graphicx}				% Use pdf, png, jpg, or eps§ with pdflatex; use eps in DVI mode
\usepackage{amsmath}								% TeX will automatically convert eps --> pdf in pdflatex		
\usepackage{amssymb}
\usepackage{cases}


%SetFonts
%SetFonts

\title{Assignment 0417}
\author{Mio Shibata}
\date{}							% Activate to display a given date or no date

\begin{document}
\maketitle
%\section{}
%\subsection{}
 
% In your main .tex file

Exercise 1-0\\
Let 
$ M=
\begin{pmatrix}
 a & b\\
 c & d
\end{pmatrix}.\\$
If given matrix of M transforms (2,3) to (2,3),\\
$ M 
\begin{pmatrix}
 2\\
 3
\end{pmatrix}
=
\begin{pmatrix}
 2\\
 3
\end{pmatrix}$
\\
Then,
\begin{numcases}
  {}
  2a + 3b = 2 & \\
  2c + 3d = 3  &
\end{numcases}
\begin{numcases}
  {}
  b = \frac{2-2a}{3} & \\
  d = \frac{3-2c}{3}  &
\end{numcases}\\
Also$(M- E)
\begin{pmatrix}
 2\\
 3
\end{pmatrix}
=
\begin{pmatrix}
 0\\
 0
\end{pmatrix}
$\\
It satisfies that
$M-E =
\begin{pmatrix}
 a-1 & b\\
 c & d-1
\end{pmatrix}$\\

By inserting (3),(4),\\
$det(M-E)=(ad-bc)+1-(a+d)\\=3a-2ac-2c+2ac+3-3a+3-2c\\=0$

Then, the factors of M satisfies this formula $ad-bc= a+d -1$.\\

Exercise 1-1\\

$ M 
\begin{pmatrix}
 p\\
 q
\end{pmatrix}
=
\begin{pmatrix}
 p\\
 q
\end{pmatrix}$\\
$(M- E)
\begin{pmatrix}
 p\\
 q
\end{pmatrix}
=
\begin{pmatrix}
 0\\
 0
\end{pmatrix}
$\\
And $(p,q) \neq 0$\\
$M-E = 0$\\
$det (M-E) = 0$\\
Then, the factors of matrix M is satisfied with this  relational expression,\\  $(ad-bc)+1-(a+d)=0$\\

Exercise 1-2\\

$ M 
\begin{pmatrix}
 2\\
 3
\end{pmatrix}
=
\begin{pmatrix}
 4\\
 6
\end{pmatrix}
 =
2
\begin{pmatrix}
 2\\
 3
\end{pmatrix}\
$\\

$(M- 2E)
\begin{pmatrix}
 2\\
 3
\end{pmatrix}
=
\begin{pmatrix}
 0\\
 0
\end{pmatrix}
$\\
Then,$M-2E = 0$\\
$det (M-2E) = 0$\\
Then, the factors of matrix M is satisfied with this  relational expression, \\ $(ad-bc)+4-2(a+d)=0$\\

Exercise 1-3\\
As I described above  Excercises1-3,  if any $(p.q)$ is not transformed by matrix of M, the factors of matrix M is satisfied with this  relational expression,\\  $(ad-bc)+1-(a+d)=0$\\
In other words, all $(p.q)$ is transformed to $(p',q')$ ($p\neq p', q\neq q'$).\\
Therefore, if $(ad-bc)+1-(a+d) \neq 0$, all $(p.q)$is transformed to others without $(p',q')\neq (0,0)$
\\



Exercise 2-1\\
Let eigenvalues of matrix of M are $\lambda 1, \lambda 2$ and each eigenvectors are $\vec v_1, \vec v_2$,\\
$\vec v_1, \vec v_2$ are described 
$ \vec v_1=
\begin{pmatrix}
 \cos {\theta_1}\\
 \sin {\theta_1} 
\end{pmatrix}$
$ \vec v_2=
\begin{pmatrix}
 \cos {\theta_2}\\
 \sin {\theta_2} 
\end{pmatrix}$.\\

$M \vec v_1 = \lambda_1 \vec v_1,M \vec v_2 =\lambda_2 \vec v_2$ \\
Let $V=(\vec v_1, \vec v_2)$, $V =\begin{pmatrix}\cos{\theta_1},\cos{\theta_2}\\ \sin{\theta_1},\sin{\theta_2} \end{pmatrix}$ 
$MV= V\begin{pmatrix}
  \lambda_1 & 0\\
 0 & \lambda_2
\end{pmatrix}$\\
$M = 
\begin{pmatrix}
 \cos{\theta_1},\cos{\theta_2}\\ 
 \sin{\theta_1},\sin{\theta_2} 
\end{pmatrix}
\begin{pmatrix}
 \lambda_1,0\\
  0,\lambda_2
  \end{pmatrix}V^{-1}\\
= \begin{pmatrix}\lambda_1\cos{\theta_1},\lambda_2\cos{\theta_2}\\ \lambda_1\sin{\theta_1},\lambda_2\sin{\theta_2} \end{pmatrix}$\\
$P^-1= \frac{1}{\cos{\theta_1}\sin{\theta_2}-\cos{\theta_2}\sin{\theta_1}}\begin{pmatrix}\sin{\theta_2}, -\cos{\theta_2}\\  -\sin{\theta_1},\cos{\theta_1} \end{pmatrix}$\\
$A = \frac{1}{\cos{\theta_1}\sin{\theta_2}-\cos{\theta_2}\sin{\theta_1}} \begin{pmatrix}\lambda_1\cos{\theta_1},\lambda_2\cos{\theta_2}\\ \lambda_1\sin{\theta_1},\lambda_2\sin{\theta_2} \end{pmatrix}\begin{pmatrix}\sin{\theta_2}, -\cos{\theta_2}\\  -\sin{\theta_1},\cos{\theta_1} \end{pmatrix}\\
= \frac{1}{\cos{\theta_1}\sin{\theta_2}-\cos{\theta_2}\sin{\theta_1}}  \begin{pmatrix}\lambda_1\cos{\theta_1}\sin{\theta_2} - \lambda_2\sin{\theta_1}\cos{\theta_1},-(\lambda_1-\lambda_2)\cos{\theta_1}\cos{\theta_2}\\ 
(\lambda_1-\lambda_2)\sin{\theta_1}\sin{\theta_2},-\lambda_1\sin{\theta_1}\cos{\theta_2}+\lambda_2 \cos{\theta_1}\sin{\theta_2}\end{pmatrix}$




Excercise 2-2\\
 
$ M=
\begin{pmatrix}
 1 & 2\\
 2 & 4
\end{pmatrix}.$\\

When one of eigenvalues is 5 and let eigenvector as $\vec x$, \\
$(M-\lambda E) \vec x = 0$\\
$(M-5 E) \vec x = 0$\\
$ (\begin{pmatrix}
 1 & 2\\
 2 & 4
\end{pmatrix}-
\begin{pmatrix}
 5 & 0\\
 0 & 5
\end{pmatrix})\vec x = 0.$\\
$\begin{pmatrix}
 -4 & 2\\
 2 & -1
\end{pmatrix})\vec x = 0$\\
then $\vec x = t\begin{pmatrix}
 1\\
 2
\end{pmatrix}$.\\

Let one coordinate on the eigenvector $\vec x$as $ (p,2p)$,\\
$\begin{pmatrix}
 1 & 2\\
 2 & 4
\end{pmatrix}
\begin{pmatrix}
 p\\
 2p
\end{pmatrix}= \begin{pmatrix}
 5p\\
 10p
\end{pmatrix}= 5\begin{pmatrix}
 p\\
 2p
\end{pmatrix}$\\


Then that coordinate is transformed to 5times of coordinates.\\ 

Exercise 2-3\\
Let one coordinate as (m,n) not on the eigenvector,\\
$\begin{pmatrix}
 1 & 2\\
 2 & 4
\end{pmatrix}
\begin{pmatrix}
 m\\
 n
\end{pmatrix}= \begin{pmatrix}
 m+2n\\
 2m+4n
\end{pmatrix}= \begin{pmatrix}
 m+2n\\
 2(m+2n)
\end{pmatrix}$\\
Then all coordinates are transformed $ t\begin{pmatrix}
 1\\
 2
\end{pmatrix}$\\

Exercise 2-4,\\
When I did solve(M) on R, the error is shown.\\
This is because $det (M) = 0$.\\
All coordinates are transformed to a line.\\
It means that if $det (M) = 0$, we can't get inverse matrix of M and transformation of matrix of M converge to a line.\\

Exercise 2-5,\\
Let $\lambda$ as eigenvalue of matrix of M and matrix  of M is given as below,\\
$M=
\begin{pmatrix}
 1 & 2\\
 3 & 4
\end{pmatrix}$\\
Then,
$det(M) =\begin{pmatrix}
 1- \lambda  & 2\\
 3 & 4- \lambda
\end{pmatrix}= 0$\\
$(1- \lambda)(4- \lambda)-2\times3 = 0$\\
$ \lambda^2 - 5 \lambda -2= 0$\\
$ \lambda =\frac{5 \pm \sqrt{33}}{2}$\\

Exercise 2-6,\\
Let matrix of $M=
\begin{pmatrix}
 a & b\\
 c & d
\end{pmatrix}$ and  $\lambda$ as eigenvalue of matrix of M,\\
$det(M) =\begin{pmatrix}
 a- \lambda  & b\\
 c & d- \lambda
\end{pmatrix}= 0$\\
$(a- \lambda)(d- \lambda)-bc = 0$\\
$ \lambda^2 - (a+d)\lambda + ad - bc= 0$\\
$ \lambda =\frac{(a+d) \pm \sqrt{a^2+d^2-2ad+4bc}}{2}$\\

Exercise 2-7,\\
From the result of Exercise 2-5,
$ \lambda =\frac{(a+d) \pm \sqrt{a^2+d^2-2ad+4bc}}{2}$\\
If eigenvalue is complex number,
$\sqrt{a^2+d^2-2ad+4bc} <0$\\
a,b,c and d are real numbers ,$a^2+d^2 \geqq 0$\\
Then,$-2ad+4bc < 0$ and $a^2+d^2<2ad-4bc$.\\

When M[1,1]=1、M[1,2]=1, and let M[2,1]=$m_1$,M[2,2]=$m_2$,\\
$M=
\begin{pmatrix}
 1 & 1\\
 m_1 & m_2
\end{pmatrix}$\\

Then let eigenvalue $\lambda$,
$(\lambda -1)( \lambda-m_2)-m_1 =0\\
\lambda^2-(m_1-1)\lambda-m_1=0$\\
Eigenvalues are complex numbers therefore $m_1 and m_2$ satisfy formula shown exercise 2-6.\\
\begin{numcases}
  {}
  -2\times 1\times m_2 + 4\times 1\times m_1 <0\\
  m_2^2 -2m_2+4m_1+1 < 0 &
\end{numcases}\\
One of the number is chosen like $m_1=-1$ and therefore $m_2^2 -2m_2-3 < 0$  $-1<m_2<3$.
Therefore , given   
$M=
\begin{pmatrix}
 1 & 1\\
 -1 & 1
\end{pmatrix}$\\
the eigenvalue is complex number.\\



Exercise 2-8,\\
When eigenvalues are two real numbers $\neq0$, each eigenvalue has eigenvector.
Therefore the transformation of matrix that has those two eigenvalues is transform coordinates to on the line whose direction vector is each eigenvector and length is determined by each eigenvalue.\\

When one eigenvalue is 0 and the other is not,
$A \vec x=0$ ($\vec x$is eigenvector)\\
Then $\vec x = 0$, meaning all transformations let all coordinates on (0,0).\\

When eigenvalues are complex number,  those eigenvalues can't describe the length in real space.\\
So if we use complex space, those eigenvalue can show the value of length in that space with eigenvectors.\\ 

Exercise 3-1,\\
To consider of $2\times 2$matrix of M,M transform $(1,0) -> (1,4), (0,1) -> (3,3)$\\ 
Let $M=
\begin{pmatrix}
 a & b\\
 c & d
\end{pmatrix}$,\\
$M\begin{pmatrix}
 1 & 0\\
 0 & 1
\end{pmatrix}=
\begin{pmatrix}
 a & b\\
 c & d\end{pmatrix}
 \begin{pmatrix}
 1 & 0\\
 0 & 1
\end{pmatrix}=\begin{pmatrix}
 1 & 3\\
 4 & 3
\end{pmatrix}$
Then $M=
\begin{pmatrix}
 1 & 3\\
 4 & 3
\end{pmatrix}$\\
Let Affine transformation as matrix of A\\
$A=
\begin{pmatrix}
 a & b & v1\\
 c & d & v2\\
 0 & 0 &1
\end{pmatrix}$\\
Each transformation by matrix of A satisfies these formula,\\
\begin{numcases}
  {}
  a + 2b + v1 = 1& \\
  c + 2d + v2 = 2  &
\end{numcases}\\
\begin{numcases}
  {}
  2a + 2b + v1 = 3& \\
  2c + 2d + v2 = 5  &
\end{numcases}\\
\begin{numcases}
  {}
  a + 3b + v1 = 4& \\
  c + 3d + v2 = 6  &
\end{numcases}\\
Then, $a=2,b=3,c=3,d=4,v1=-7 ,v2 = -9$.
$A=
\begin{pmatrix}
 2 & 3 & -7\\
 3 & 4 & -9\\
 0 & 0 &1
\end{pmatrix}$\\
To check the transformation of (1,2) by A is (1,2),\\
$A=
\begin{pmatrix}
 2 & 3 & -7\\
 3 & 4 & -9\\
 0 & 0 &1
\end{pmatrix}
\begin{pmatrix}
 1\\
 2\\
 0
\end{pmatrix}\\
 =\begin{pmatrix}
 1\\
 2\\
 0
\end{pmatrix}$




%$M\begin{pmatlix}
%1 \\ 0 \end{pmatlix}=\begin{pmatlix} \\c \end{pmatlix}$

%$M\begin{pmatlix}0\\1\end{pmatlix}=\begin{pmatlix}\\d\end{pmatlix}$


\end{document}  